
\documentclass[12pt]{article}

\usepackage[utf8]{inputenc}
\usepackage[T2A]{fontenc}
\usepackage[english, russian]{babel}

\usepackage[pdftex]{graphicx}
\usepackage{indentfirst}
\usepackage{float}
\usepackage{wrapfig}
\usepackage{pdflscape}
\usepackage{amssymb,amsfonts,amsmath}
\usepackage{tikz}
\usepackage{xspace}
\usepackage{changepage}
\usepackage{multicol}

\setcounter{tocdepth}{7}

\usepackage{listingsutf8}
\lstloadlanguages{C}
\lstset
{
        language = C,
		breaklines,
		columns = fullflexible,
		flexiblecolumns,
		numbers = none,
        basicstyle = \tt\fontsize{12pt}{12pt}\selectfont,
        commentstyle = ,
        showtabs = false, 
        showspaces = false,
        showstringspaces = false,
        tabsize = 2,
        inputencoding = utf8/cp1251,
		frame = single,
		showlines = true,
		resetmargins = true
}

\usepackage{geometry}
\geometry{left=2cm}
\geometry{right=2cm}
\geometry{top=2cm}
\geometry{bottom=2cm}

\renewcommand{\theenumi}{\arabic{enumi}.}
\renewcommand{\theenumii}{\arabic{enumii}}
\renewcommand{\theenumiii}{.\arabic{enumiii}}
\renewcommand{\theenumiv}{.\arabic{enumiv}}

\renewcommand{\labelenumi}{\arabic{enumi}.}
\renewcommand{\labelenumii}{\arabic{enumi}.\arabic{enumii}}
%\renewcommand{\labelenumiii}{\arabic{enumi}.\arabic{enumii}.\arabic{enumiii}}
%\renewcommand{\labelenumiv}{\arabic{enumi}.\arabic{enumii}.\arabic{enumiii}.\arabic{enumiv}}

% Переопределение \caption
\makeatletter
\renewcommand{\@biblabel}[1]{#1.}
\renewcommand{\@makecaption}[2]{%
\vspace{\abovecaptionskip}%
\sbox{\@tempboxa}{#1: #2}
\ifdim \wd\@tempboxa >\hsize
   #1: #2\par
\else
   \global\@minipagefalse
   \hbox to \hsize {\hfil \small Рисунок \thefigure~<<#2>>\hfil}%
\fi
\vspace{\belowcaptionskip}}
\makeatother

\newcommand{\mysection}[1]
{
	\clearpage
	\section{#1}
}

\newcommand{\mysectionvn}[1]
{
	\clearpage
	\section*{#1}
}

\newcommand{\mysubsection}[1]
{
	\subsection{#1}
}

\newcommand{\mysubsectionvn}[1]
{
	\subsection*{#1}
}

\newcommand{\mysubsubsection}[1]
{
	\subsubsection{#1}
	\setcounter{paragraph}{0}
	\setcounter{subparagraph}{0}
}

\newcommand{\myparagraph}[1]
{
	\medskip
	\refstepcounter{paragraph}
	{\raggedleft\bf\theparagraph\quad#1}
	\addcontentsline{toc}{paragraph}{\theparagraph\quad#1}
	\medskip
	\setcounter{subparagraph}{0}
}

\newcommand{\myparagraphvn}[1]
{
	\medskip
	\refstepcounter{paragraph}
	{\bf\quad#1}
	\medskip
}

\newcommand{\mysubparagraph}[1]
{
	\medskip
	\refstepcounter{subparagraph}
	{\raggedleft\bf\thesubparagraph\quad#1}
	\addcontentsline{toc}{subparagraph}{\thesubparagraph\quad#1}
	\medskip
}

\newcommand{\mysubparagraphvn}[1]
{
	\medskip
	\refstepcounter{subparagraph}
	{\bf\quad#1}
	\medskip
}

\makeatletter
\renewcommand{\tableofcontents}
{
	\clearpage
	\par{\bf \Large \noindent \centerline{Содержание}}
	\par
	\@starttoc{toc}
}
\makeatother

\makeatletter 
\renewcommand\appendix
{
        \par
        \setcounter{section}{0}
        \gdef\thesection{\@Asbuk\c@section}
}
\makeatother

\newcommand\myappendix[1]
{
		\clearpage
        \refstepcounter{section}
        
        \section*{Приложение~\thesection.~#1}
        
        \addcontentsline{toc}{section}{Приложение~\thesection.~#1}
}

\newcommand{\mysource}[3]
{
        \refstepcounter{lstcon}
		\label{listing:#2}
		{
			\lstinputlisting[]{#1}
			\nopagebreak
			
			\vbox{\small \centering Листинг~\thelstcon~---~#3}
			\bigskip
		}
}

\newcommand{\mysourcepart}[5]
{
        \refstepcounter{lstcon}
		\label{listing:#3}
		{
			\lstinputlisting[#5]{#1#2}
			\nopagebreak
			
			\vbox{\small \centering Листинг~\thelstcon~---~Функция~<<#4>>}
			\bigskip
		}
}

% Листинги внутри локумента
%
% \mylistingbegin{Метка в listing:}{Подпись}
% \begin{lstlisting}
% \end{lstlisting}
% \mylistingend
%

\newcommand{\mylistingcaption}{}
\newcommand{\mylistinglabel}{}

\newcommand{\mylistingbegin}[2]
{
	\refstepcounter{lstcon}
	\renewcommand{\mylistingcaption}{\vbox{\small \centering Листинг~\thelstcon~-~#2}}
	\renewcommand{\mylistinglabel}{\label{listing:#1}}
	\begin{adjustwidth}{-\leftmargin}{\rightmargin}
}

\newcommand{\mylistingend}
{
	\mylistingcaption
	\mylistinglabel
	\end{adjustwidth}
	\medskip
}

\newcommand{\mimage}[4]
{
	\vbox
	{
		\begin{center}

			\refstepcounter{figure}
			\label{image:#1}
			
			\includegraphics[#4]{image/#2}

			\medskip
			{\noindent \small Рисунок~\thefigure~-~#3}

		\end{center}
	}
}

\renewcommand{\thefigure}{\arabic{figure}}
\setcounter{figure}{0}

\newcounter{lstcon}
\setcounter{lstcon}{0}

\renewcommand{\theenumi}{} % \asbuk{enumi}}
\renewcommand{\theenumii}{\arabic{enumii}}
\renewcommand{\theenumiii}{\arabic{enumiii}}
\renewcommand{\labelenumi}{\arabic{enumi})}
\renewcommand{\labelenumii}{\arabic{enumi}.\arabic{enumii})}
\renewcommand{\labelenumiii}{\arabic{enumi}.\arabic{enumii}.\arabic{enumiii})}
% \renewcommand{\labelenumi}{\asbuk{enumi})} % ГОСТы
% \renewcommand{\labelenumii}{\arabic{enumii})}

\renewcommand{\labelitemi}{$-$}
\renewcommand{\labelitemii}{$-$}

\makeatletter
\def\tableofcontents{\mysectionvn{Содержание}\@starttoc{toc}}
\makeatother

\makeatletter
\renewcommand{\@biblabel}[1]{#1.}
\makeatother

\newcommand{\gl}{ОС GNU/Linux\xspace}



\newcounter{lr}
\setcounter{lr}{1}

\newcommand{\lr}[1]
{
	\mysection{#1}
	
		\mysubsection{Цель работы}
		
Ознакомится с принципами организации параллельных вычислений в многопроцессорных системах с распределенной памятью. Ознакомится с принципами организации многопоточных приложений, использующих пересылку сообщений между вычислительными потоками как средство обмена данными между оными. Получить навыки построения вычислительных кластеров. Получить навыки параллельного программирования для многопроцессорных систем с распределенной памятью с помощью языка программирования C, \gl и библиотеки OpenMPI, реализующей стандарт MPI.



		\mysubsection{Теоретическая часть}
		
\mysubsubsection{Введение}

В предыдущей лабораторной работе были рассмотрены принципы организации параллельных и распределенных вычислений в многопроцессорных системах с разделяемой памятью. Вычислительные потоки в данных системах взаимодействуют друг с другом с помощью областей оперативной памяти, одинаково доступных каждому из потоков. Таковой способ взаимодействия вычислительных потоков чреват возникновением большого числа явных или скрытых ошибок, связанных с одновременным доступом к одной и той же ячейке памяти несколькими потоками. Кроме того, таковой способ взаимодействия вычислительных потоков очевидно не может быть реализован в тех ситуациях, когда необходимо реализовать параллельные или распределенные вычисления с помощью вычислительных кластеров - с помощью многопроцессорных систем с распределенной памятью.

Для реализации параллельных и распределенных вычислений с помощью многопроцессорных систем с распределенной памятью используют модель вычислительных потоков, взаимодействующих друг с другом путем пересылки сообщений. Таковой способ взаимодействия вычислительных потоков позволяет программисту избавится от ошибок, связанных с параллельным доступом вычислительных потоков к разделяемой памяти, и объединить некоторые задачи - например, ожидание вычислительным потоком поступления данных от другого вычислительного потока и чтение этих данных.

\mysubsubsection{Многопоточное программирование в \gl с взаимодействием вычислительных потоков путем обмена сообщениями}

Как и в случае взаимодействия вычислительных потоков через разделяемую память, самый низкоуровневый (и доступный во всех установка \gl) способ организации многопоточных вычислений с взаимодействием вычислительных потоков путем обмена сообщениями в \gl основан на использовании системных вызовов.

Помимо системных вызовов fork, wait, waitpid, wait4, использующихся для создания процессов системы, реализующих вычислительные потоки, и ожидания завершения процессов системы, в таковых многопоточных приложения необходимо использовать системные вызовы socket, bind, accept, send, recv, sendto, recvfrom, close - словом, системные вызовы, позволяющие организовать взаимодействие (удаленных) процессов с помощью сокетов. Для взаимодействия процессов могут использоваться как сокеты типа AF\_UNIX, позволяющие организовать взаимодействие процессов, запущенных в одной и той же системе, так и типов AF\_INET, AF\_INET6 и прочие, позволяющие организовать взаимодействие процессов, запущенных в различных, удаленных друг от друга, вычислительных системах (сокеты типа AF\_INET используют протоколы транспортного уровня стека протоколов TCP/IPv4, сокеты AF\_INET6 - протоколы транспортного уровня стека протоколов TCP/IPv6). В некоторых вычислительных системах (к которым относятся современные суперкомпьютеры) взаимодействие процессов может основываться на более быстрой технологии, чем протоколы TCP, SCTP и UDP. В вычислительных кластерах, состоящих из нескольких персональных компьютеров, объединенных в вычислительную сеть, построенную, например, с помощью технологии Ethernet, использование протоколов TCP, SCTP и UDP представляется наиболее целесообразным - и именно сетевое взаимодействие становится наиболее затратным по времени в таковых вычислительных кластерах (как следствие, возникает необходимость минимизировать количество пересылок сообщений между вычислительными потоками и минимизировать объемы пересылаемых сообщений).

С точки зрения программиста, организация параллельных и распределенных вычислений с взаимодействием вычислительных потоков путем пересылки сообщений с помощью системных вызовов \gl является значительно более трудоемкой задачей по сравнению с организацией параллельных и распределенных вычислений с взаимодействием вычислительных потоков, взаимодействующих через разделяемую память, по следующим причинам:

\begin{itemize}

	\item Программист должен самостоятельно реализовать собственный протокол обмена сообщениями поверх протоколов транспортного уровня стека протоколов TCP/IP;
	\item Программист должен самостоятельно отслеживать все ошибки, могущие возникнуть в процессе сетевого обмена;
	\item Программист должен самостоятельно реализовать пересылку исполняемого кода на удаленные системы;
	\item Программист должен самостоятельно реализовать функционал запуска исполняемого кода на удаленных системах.

\end{itemize}

К счастью, на настоящий момент у программиста имеется несколько способов существенно упростить процесс организации параллельных и распределенных вычислений:

\begin{itemize}

	\item Использование специальных библиотек (например, библиотек, реализующих стандарт MPI - библиотеки MPICH, OpenMPI и другие);
	\item Использование языков параллельного программирования (например, язык программирования Erlang).

\end{itemize}

В настоящей лабораторной работе рассматривается способ организации параллельных и распределенных вычислений с взаимодействием вычислительных потоков путем обмена сообщениями с помощью библиотеки OpenMPI, реализующей стандарт MPI.

\mysubsubsection{Стандарт MPI. Библиотека OpenMPI}

Стандарт MPI (Message Passing Interface, интерфейс обмена сообщениями) был разработан группой ученых во главе с профессором Иллинойского университета Уильямом Гроуппом в 1992. В настоящее время стандарт MPI дорабатывается некоммерческим объединением <<MPI Forum>> \cite{mpi-forum}. Существуют версии стандарта MPI для языков программирования C, C++, Fortran.

В настоящее время имеется несколько реализаций стандарта MPI:

\begin{itemize}

	\item Библиотека MPICH;
	\item Библиотека OpenMPI \cite{openmpi};
	\item Библиотека Intel MPI;
	\item Библиотека HP MPI;
	\item Прочие библиотеки.

\end{itemize}

В настоящей лабораторной работе используется реализация стандарта MPI для языка программирования C из состава библиотеки OpenMPI. В качестве компилятора для языка программирования C в настоящей лабораторной работе используется компилятор GNU C Compiler из состава коллекции компиляторов GNU Compiler Collection.

Стандарт MPI позволяет программисту реализовать параллельные вычисления в несколько потоков. Вычислительные потоки объединяются в группы потоков (для простоты будем рассматривать группу потоков, идентифицируемую константой компилятора \linebreak MPI\_COMM\_WORLD - вопросы разделения потоков на группы, объединение групп потоков и прочие вопросы, связанные с группировкой потоков, выходят за рамки лабораторной работы). Потоки нумеруются, считая от 0 (нулевой поток - главный поток).

Каждый вычислительный поток получает для выполнения один и тот же бинарный код, разграничение выполняемого потоками кода может быть реализовано с помощью условных операторов и проверки номера потока.

В начале выполняемого потоками кода должен находится вызов функции {\bf MPI\_Init()}, прототип которой приведен в листинге \ref{listing:mpi-init}. Функция MPI\_Init() инициализирует параллельную часть вычислительного потока.

\mylistingbegin{mpi-init}{Функция MPI\_Init()}
\begin{lstlisting}

int MPI_Init(int * argc, char *** argv);

\end{lstlisting}
\mylistingend

Функция MPI\_Init() обладает следующими параметрами:

\begin{itemize}

	\item argc - указатель на переменную, хранящую количество аргументов, переданных вычислительному потоку при его запуске;
	\item argv - указатель на массив строк, хранящий аргументы, переданные вычислительному потоку при его запуске.

\end{itemize}

Значения параметров argc и argv функции MPI\_Init() должны быть равны адресам параметров argc и argv функции main() вычислительного потока.

Функция MPI\_Init() возвращает значение константы компилятора MPI\_SUCCESS в случае своего успешного завершения и значение, отличное от значения указанной константы компилятора, в случае неудачного своего завершения. То же возвращают и все прочие функции стандарта MPI, рассмотренные в настоящей лабораторной работе, если только не указано обратное.

По завершению параллельной части своего кода вычислительный поток должен вызвать функцию {\bf MPI\_Finalize()}, прототип которой приведен в листинге \ref{listing:mpi-finalize}.

\mylistingbegin{mpi-finalize}{Функция MPI\_Finalize()}
\begin{lstlisting}

int MPI_Finalize();

\end{lstlisting}
\mylistingend

Вычислительный поток может получить свой номер, вызвав функцию \linebreak {\bf MPI\_Comm\_rank()}, прототип которой приведен в листинге \ref{listing:mpi-comm-rank}. Для получения же числа потоков в определенной группе потоков вычислительный поток может воспользоваться функцией {\bf MPI\_Comm\_size()}, прототип которой приведен в листинге \ref{listing:mpi-comm-size}.

\mylistingbegin{mpi-comm-rank}{Функция MPI\_Comm\_rank()}
\begin{lstlisting}

int MPI_Comm_rank(MPI_COMM_WORLD, int * rank);

\end{lstlisting}
\mylistingend

\mylistingbegin{mpi-comm-size}{Функция MPI\_Comm\_size()}
\begin{lstlisting}

int MPI_Comm_size(MPI_COMM_WORLD, int * size);

\end{lstlisting}
\mylistingend

Функция MPI\_Comm\_rank() присвоит номер потока переменной, на которую указывает ее параметр rank. Функция MPI\_Comm\_size() присвоит количество потоков в группе MPI\_COMM\_WORLD переменной, на которую указывает ее параметр size.

Для отправки и приема сообщений вычислительные потоки могут использовать следующий функционал стандарта MPI:

\begin{itemize}

	\item Функции {\bf MPI\_Send()} и {\bf MPI\_Recv()}.

	Функция MPI\_Send() позволяет вычислительному потоку отправить сообщение целевому вычислительному потоку или всем вычислительным потокам, входящим в целевую группу потоков. Прототип функции MPI\_Send() приведен в листинге \ref{listing:mpi-send}.

\mylistingbegin{mpi-send}{Функция MPI\_Send()}
\begin{lstlisting}

int MPI_Send(void * buf, int count, MPI_Datatype type, int dst, int tag, MPI_COMM_WORLD);

\end{lstlisting}
\mylistingend

	Функция MPI\_Send() отправляет count элементов типа type из буфера buf. В качестве значения параметра type могут использоваться значения следующих констант компилятора:

	\begin{itemize}

		\item MPI\_CHAR - char;
		\item MPI\_SHORT - short int;
		\item MPI\_INT - int;
		\item MPI\_LONG - long int;
		\item MPI\_UNSIGNED\_CHAR - unsigned char;
		\item MPI\_UNSIGNED\_SHORT - unsigned short int;
		\item MPI\_UNSIGNED - unsigned int;
		\item MPI\_UNSIGNED\_LONG - unsigned long int;
		\item MPI\_FLOAT - float;
		\item MPI\_DOUBLE - double;
		\item MPI\_LONG\_DOUBLE - long double.

	\end{itemize}

	Таким образом, функция MPI\_Send() отправит целевому вычислительному потоку (потокам) первые (count * sizeof(type)) байт из буфера buf потока - отправителя.

	Значение параметра dst функции MPI\_Send() суть есть номер того вычислительного потока, которому отправляется сообщение. Для широковещательной рассылки сообщений в качестве значения параметра dst необходимо указать значение константы компилятора MPI\_ANY\_SOURCE.

	Значение параметра tag указывает тег сообщения. Тег сообщение - положительное целое число, идентифицирующее данное сообщение (тип данного сообщения).

	Поток - супервизор помещает сообщение, отправленное некоторому вычислительному потоку, в очередь сообщений данного потока, из которого оно может быть выбрано потоком с помощью функции MPI\_Recv(). При этом соблюдается временная очередность выборки сообщений из очереди - первым будет выбрано то сообщение от указанного источника с указанным тегом, которое было помещено в очередь раньше остальных. Прототип функции MPI\_Recv() приведен в листинге \ref{listing:mpi-recv}.

\mylistingbegin{mpi-recv}{Функция MPI\_Recv()}
\begin{lstlisting}

int MPI_Recv(void * buf, int count, MPI_Datatype type, int src, int tag, MPI_COMM_WORLD, MPI_Status * status);

\end{lstlisting}
\mylistingend

	Функция MPI\_Recv() принимает сообщение с тегом tag от вычислительного потока с номером src, помещая count элементов типа type сообщения в буфер buf. В переменную, на которую указывает параметр status, помещается информация о параметрах принятого сообщения.

	В случае, если сообщения с указанным тегом от указанного источника в очереди сообщений отсутствует, вычислительный поток блокируется в ожидании такового сообщения.

	Для приема сообщений с любым тегом в качестве значения параметра tag необходимо указать значение константы компилятора MPI\_ANY\_TAG.

	Для приема сообщений от любого источника в качестве значения параметра src необходимо указать значение константы компилятора MPI\_ANY\_SOURCE;

	\item Функции {\bf MPI\_Bcast()} и {\bf MPI\_Gather()}.

	Функция MPI\_Bcast() позволяет организовать широковещательную рассылку сообщения. Прототип функции MPI\_Bcast() приведен в листинге \ref{listing:mpi-bcast}.

\mylistingbegin{mpi-bcast}{Функция MPI\_Bcast()}
\begin{lstlisting}

int MPI_Bcast(void * buf, int count, MPI_Datatype type, int src, MPI_COMM_WORLD);

\end{lstlisting}
\mylistingend

	Функция MPI\_Bcast() отправляет count элементов типа type из буфера buf все процессам из группы процессов MPI\_COMM\_WORLD.

	Номер потока, отправляющего сообщение, указывается параметром src. Данный поток также получит отправляемое им сообщение.

	Функция MPI\_Gather() позволяет организовать сбор данных со всех вычислительных потоков одним из них. Прототип функции MPI\_Gather() приведен в листинге \ref{listing:mpi-gather}.

\mylistingbegin{mpi-gather}{Функция MPI\_Gather()}
\begin{lstlisting}

int MPI_Gather(void * src_buf, int src_count, MPI_Datatype src_type, void * dst_buf, int dst_count, MPI_Datatype dst_type, int dst, MPI_COMM_WORLD);

\end{lstlisting}
\mylistingend

	Вычислительный поток с номером dst принимает сообщения от всех вычислительных потоков группы MPI\_COMM\_WORLD (в том числе, и от себя самого).

	Каждый вычислительный поток отправляет src\_count элементов типа src\_type из буфера src\_buf.

	Вычислительный поток с номером dst принимает по dst\_count элементов типа dst\_type от каждого вычислительного потока и помещает принятые данные в буфере dst\_buf, располагая блоки данных в порядке возрастания номеров отправивших их вычислительных потоков.

\end{itemize}

При разработке многопоточных приложений, использующих стандарт MPI, полезным будет также использование функции {\bf MPI\_Barrier()}, позволяющей синхронизировать выполнение вычислительных потоков. Прототип функции MPI\_Barrier() приведен в листинге \ref{listing:mpi-barrier}.

\mylistingbegin{mpi-barrier}{Функция MPI\_Barrier()}
\begin{lstlisting}

int MPI_Barrier(MPI_COMM_WORLD);

\end{lstlisting}
\mylistingend

Вычислительный поток, входящий в группу потоков MPI\_COMM\_WORLD, блокируется функцией MPI\_Barrier() до тех пор, пока все остальные потоки данной группы не вызовут функцию MPI\_Barrier();

\mysubsubsection{Сборка и запуск программ, использующих библиотеку OpenMPI}

В исходный код программы, использующей библиотеку OpenMPI для организации параллельных или распределенных вычислений, необходимо включить заголовочный файл mpi.h.

Для сборки программы, использующей библиотеку OpenMPI для организации параллельных или распределенных вычислений, с помощью компилятора GNU C Compiler, необходимо использовать оболочку к означенному компилятору, предоставляемую библиотекой OpenMPI. Данная оболочка реализована в виде исполняемого файла mpicc. Исполняемый файл mpicc может быть запущен с аргументами, передаваемыми исполняемому файлу gcc компилятора GNU C Compiler - аргументы, необходимые для корректного связывания программы с библиотекой OpenMPI, исполняемый файл mpicc добавит самостоятельно в процессе своего выполнения.

Процесс запуска программы, использующей библиотеку OpenMPI, на вычислительном кластере состоит из следующих шагов:

\begin{enumerate}

	\item Настройка среды запуска\footnote{В лаборатории настройка среды запуска выполняется лаборантами.}.

	Вычислительные потоки, запускаемые на отдельных вычислительных системах, входящих в состав кластера, обмениваются между собой сообщениями по тем или иным протоколам.

	Существуют протоколы обмена сообщениями, специально разработанные (и, следовательно, оптимизированные) для организации вычислительных кластеров. В случае же организации кластера на основе нескольких персональных компьютеров, объединенных друг с другом в вычислительную сеть, в качестве протоколов обмена сообщениями используются следующие протоколы:

	\begin{itemize}

		\item Протокол RSH;
		
		\item Протокол SSH.

			Протокол SSH дополнительно применяет шифрование для защиты пересылаемых данных, что негативно сказывается на временных характеристиках процесса обмена сообщениями между вычислительными потоками.

	\end{itemize}

	Следует помнить, что программа будет запущена на удаленных вычислительных системах от имени того же пользователя, от имени которого она запускается в головной вычислительной системе кластера. Таким образом, данный пользователь на удаленных вычислительных системах должен существовать и ему должно быть разрешено удаленное подключение без ввода пароля (в случае протокола SSH - аутентификация по ключу);

	\item Сборка программы;

	\item Рассылка исполняемого файла программы на удаленные вычислительные системы.

	Исполняемый файл программы может быть передан удаленным вычислительным системам следующими способами:

	\begin{itemize}

		\item Передачей на съемном носителе;
		\item Передачей с помощью утилиты ncat (nc; при этом требуется удаленный доступ на целевую вычислительную систему по протоколам RSH или SSH);

	\end{itemize}

	\item Запуск программы.

	Для запуска программы используется утилита mpirun, входящая в состав библиотеки OpenMPI. Утилита mpirun принимает следующие аргументы:

	\begin{itemize}

		\item -mca orte\_rsh\_agent rsh - использование для обмена сообщениями протокола RSH;
		\item -mca orte\_rsh\_agent ssh - использование для обмена сообщениями протокола SSH;
		\item -mca orte\_rsh\_agent "ssh : rsh" - использование для обмена сообщениями протокола SSH или, если подключение по SSH завершилось неудачей, протокола RSH;
		\item -n NUM - предписание запустить NUM вычислительных потоков;
		\item -H IP\_1,IP\_2,...,IP\_N - объединить в вычислительный кластер удаленные вычислительные системы с перечисленными IP-адресами;
		\item -loadbalance - равномерное распределение вычислительных потоков по удаленным вычислительным системам (на одной вычислительной системе может быть запущено несколько вычислительных потоков);
		\item -x PATH=\verb|"|/bin:/usr/bin:/sbin:/usr/sbin:/usr/local/bin:/usr/local/sbin:DIR\verb|"| - предписание запустить программу с указанным значением переменной окружения PATH.

		Здесь DIR - абсолютный путь к каталогу, содержащему исполняемый файл программы, на удаленной вычислительной системе. Если исполняемый файл программы находится в разных каталогах на разных вычислительных системах, то необходимо продолжить перечисление через двоеточие.

		На удаленных вычислительных системах исполняемый файл программы будет запущен с помощью утилиты orted, поэтому принципиально важно нахождение ее исполняемого файла в каталогах, перечисленных в переменной окружения PATH;

		\item Имя исполняемого файла программы.

		Исполняемый файл программы должен находится в одном из каталогов, перечисленных в переменной PATH.

	\end{itemize}

	Стандартные потоки вывода и ошибок каждого вычислительного потока будут переправлены в тот терминал, в котором программа была запущена на выполнение на кластере с помощью утилиты mpirun.

\end{enumerate}


		
		\mysubsection{Демонстрационный пример}
		
% ROOT / Proof

TODO ROOT / Proof example


		
		\mysubsection{Требования к оформлению отчета по лабораторной работе}
		
Отчет по лабораторной работе должен содержать:

\begin{itemize}

	\item титульный лист, оформленный в соответствии с требованиями преподавателя, кафедры и университета и действующих государственных стандартов по оформлению титульных листов отчетов по лабораторным работам;
	
	\item задание к лабораторной работе;

	\item исходный код разработанного программного модуля.

	Исходный код программного модуля должен быть написан на языке программирования Erlang;

	\item снимки экрана, иллюстрирующие процессы запуска виртуальных машин на вычислительных узлах и процессы компиляции разработанного программного модуля на данных виртуальных машинах.

	Количество вычислительных узлов должно быть не меньше двух;

	\item снимок экрана, иллюстрирующий запуск программы на наборе данных, предполагающем небольшую временную сложность вычислений, и результаты выполнения ручных расчетов для данного набора.

	Для ручного расчета разрешается применять специализированное программное обеспечение - Octave, Sage, Maxima и прочее;

	\item снимки экрана, иллюстрирующие, как минимум, три запуска программы на наборах данных, соответствующих заданию к лабораторной работе;

	\item вывод.

	Вывод по выполнению лабораторной работы должен содержать, кроме всего прочего, заключение об эффективности (временном выигрыше или отсутствии такового) распараллеливания реализуемого алгоритма обработки исходных данных.

\end{itemize}



	\addtocounter{lr}{1}
}

\begin{document}

	
\begin{titlepage}

\begin{center}

Министерство образования и науки Российской Федерации \\
Рязанский Государственный Радиотехнический Университет

\bigskip

Кафедра ЭВМ

\vspace{5em}

В.А. Саблина, М.В. Акинин

\vspace{3em}

{\bf TODO Название МУ}

\bigskip

Методические указания\\
к лабораторным работам\\
по курсу <<TODO название курса>>

\vfill

TODO
% \includegraphics{image/title.eps}

\vfill

Рязань \the\year

\end{center}

\thispagestyle{empty}

\end{titlepage}

\setcounter{page}{2}

\newpage


	
	\clearpage
	
Содержат теоретические сведения и практические задания по лабораторным работам по дисциплине <<Параллельное программирование>>.


	\clearpage

	\tableofcontents

	\mysectionvn{Регламент проведения лабораторных работ}
	\addcontentsline{toc}{section}{Регламент проведения лабораторных работ}

		
Для успешного выполнения и сдачи лабораторных работ в отведенные для этого учебной программой сроки студент должен соблюдать следующие правила:

\begin{enumerate}

	\item Задания к лабораторным работам:

	\begin{enumerate}
		
		\item Каждая бригада студентов получает индивидуальное задание, номер которого равен увеличенному на единицу остатку от деления на 10 номера бригады студентов в журнале лабораторных работ;
		\item Студенты с хорошей успеваемостью, желающие выполнить более сложное задание, могут выбрать его себе из списка заданий повышенной сложности. Задания повышенной сложности отмечены символом *. В случае такового выбора студент освобождается от выполнения задания, выданного его бригаде. Не допускается бригадное выполнение задания повышенной сложности;
		\item Единожды полученные задания выполняются бригадами студентов / студентами на каждой из лабораторных работ каждый раз в установленных методическими указаниями (МУ) к лабораторной работе рамках с использованием программных средств и языка программирования, указанных в означенных МУ;
		\item Допускается отказ от выполнения студентом задания повышенной сложности. В этом случае он обязан выполнять задание своей бригады;
		\item Не допускается замена заданий;
		\item Качество заданий, полнота их описания и прочие вопросы, относящиеся к заданиям и определенные преподавателем, не обсуждаются. Отсутствие теоретических рекомендаций по выполнению задания предполагает самостоятельное изучение вопроса;

	\end{enumerate}

	\item Количество выполняемых лабораторных работ устанавливается преподавателем {\bf в начале семестра} и не может быть пересмотрено по просьбе студентов группы;

	\item Схема проведения лабораторной работы:

	\begin{enumerate}

		\item Допуск студентов к выполнению лабораторной работы;
		\item Выполнение студентами лабораторной работы;\label{list:exec}
		\item Демонстрация преподавателю результатов выполнения лабораторной работы;
		\item Оценка преподавателем качества выполнения лабораторной работы. В случае неудовлетворительной оценки выполняется переход к пункту \ref{list:exec};
		\item Подготовка {\bf каждым} студентом индивидуального отчета по лабораторной работе;\label{list:defend}
		\item Допуск студентов к защите лабораторной работы;
		\item Защита студентами лабораторной работы. В случае провальной защиты лабораторной работы выполняется переход к пункту \ref{list:exec} или к пункту \ref{list:defend} по выбору преподавателя;

	\end{enumerate}

	\item Выполнение лабораторных работ:

	\begin{enumerate}

		\item К выполнению лабораторной работы допускаются студенты, успешно выполнившие все предыдущие лабораторные работы и успешно защитившие все предыдущие, кроме, возможно, последней, лабораторные работы;
		\item В случае, если в бригаде до выполнения лабораторной работы допущены лишь часть студентов, то студенты данной бригады, не допущенные к лабораторной работе, должны будут {\bf самостоятельно} выполнить практическую часть лабораторной работы и продемонстрировать результаты ее выполнения преподавателю;
		\item После того, как лабораторная работа была выполнена, результаты ее выполнения должны быть продемонстрированы преподавателю;

	\end{enumerate}

	\item Защита лабораторных работ:

	\begin{enumerate}

		\item Для допуска к защите лабораторных работ {\bf каждый} студент группы должен самостоятельно подготовить уникальный отчет;
		\item Для допуска к защите лабораторных работ студент должен защитить все предыдущие лабораторные работы;
		\item Отчет должен соответствовать требованиям, перечисленным в настоящей МУ, и личным требованиям преподавателя;
		\item За преподавателем остается право не допустить студента до защиты лабораторной работы или аннулировать данный допуск в следующих случаях:

		\begin{enumerate}

			\item Студент не выполнил лабораторную работу или не продемонстрировал преподавателю в достаточной степени\footnote{<<Достаточность>> здесь и везде определяется преподователем.} работоспособность разработанного им программного решения;
			\item У преподавателя имеются сомнения в самостоятельном выполнении студентом лабораторной работы;
			\item У преподавателя имеются сомнения в самостоятельной подготовке студентом отчета по лабораторной работе;
			\item Во всех прочих случаях, когда преподаватель по той или иной причине не считает возможным допустить студента до защиты лабораторной работы;

		\end{enumerate}

		\item Для успешной защиты лабораторной работы студент должен в достаточной степени полно и правильно ответить на все вопросы преподавателя. Преподаватель имеет право задавать любой вопрос по отчету, методическим указаниям к защищаемой или предыдущим лабораторным работам, лекциям. Для решения спорных вопросов относительно присутствия того или иного материала в прочитанных на данный момент лекциях следует обращаться к преподавателям, читавшим (читающим) данные курсы. Конспект студента не является аргументом в споре о правильности того или иного решения преподавателя;

	\end{enumerate}

	\item Отчет по лабораторной работе:

	\begin{enumerate}

		\item Отчет по лабораторной работе должен быть представлен в печатном виде;
		\item Отчет по лабораторной работе должен быть выполнен в соответствии с требованиями преподавателя, кафедры, университета и в соответствии с действующими на данный момент государственными стандартами;
		\item Отчет должен быть подготовлен, по возможности, с помощью свободного программного обеспечения, к коему относятся офисный пакет LibreOffice, текстовый процессор AbiWord, система подготовки текстов к публикации \LaTeX~и прочее программное обеспечение;
		\item Иллюстрации в отчетах должны быть выполнены, по возможности, с помощью свободного программного обеспечения, к коему относятся редакторы диаграмм Dia и LibreOffice Draw, векторные графические редакторы Inkscape и Sk1, растровые графические редакторы Gimp, Krita и Nip2 и прочее программное обеспечение;
		\item Факт подготовки отчета с помощью проприетарного программного обеспечения (например, с помощью операционной системы Windows или (и) с помощью офисного пакета Microsoft Office) является отягчающим обстоятельством при принятия решения о допуске студента до защиты лабораторной работы в случае низкого качества отчета;
		\item Не допускаются до защиты отчеты, выполненные (написанные и проиллюстрированные) полностью или частично от руки;
		\item Факт наличия большого количества исправлений, внесенных от руки, орфографических и пунктационных ошибок в отчете является отягчающим обстоятельством при принятия решения о допуске студента до защиты лабораторной работы в случае низкого качества отчета;

	\end{enumerate}

	\item За преподавателем остается право не аргументировать любое свое решение;
	\item Настоящие правила не обсуждаются;
	\item Студент считается ознакомленным с правилами после того, как им было начато выполнение первой лабораторной работы;
	\item Настоящие правила могут быть {\bf дополнены} преподавателем положениями, не отменяющими полностью или частично перечисленные положения. Преподаватель обязан незамедлительно (на текущем или следующем занятии) довести до сведения студентов изменения в правилах.

\end{enumerate}



	\mysectionvn{Задания к лабораторным работам}
	\addcontentsline{toc}{section}{Задания к лабораторным работам}

		
\begin{enumerate}

	\item Умножение матриц.
	
		Вычислить матрицу $R$ такую, что $R ~=~ AB$. Здесь $A$ и $B$ матрицы размером не менее 20-ти строк и 20-ти столбцов каждая. Размеры матриц $A$ и $B$ должны указываться оператором программы в процессе ее выполнения;

	\item Решение системы линейных уравнений методом Гаусса.

		Количество уравнений в системе должно быть не менее 20-ти. Демонстрация выполнения задания должна проводится на совместной невырожденной системе линейных уравнений, однако программа должна определять и корректно обрабатывать случаи несовместной или (и) вырожденной системы;

	\item Расчет определителя квадратной матрицы.

		Количество строк в матрице должно быть не менее 7-и;

	\item Расчет определенного интеграла функции от одной переменной.

		Пределы интегрирования должны указываться оператором программы в процессе ее выполнения. Интегрируемую функцию следует реализовать как отдельную подпрограмму, принимающую единственный параметр типа \verb|double| - значение переменной, и возвращающую значение типа \verb|double| - результат вычисления функции. Не допускается программная реализация неопределенного интеграла от интегрируемой функции;

	\item Расчет матрицы ковариации случайного вектора размерности 3.

		Первая компонента реализации случайного вектора должна генерироваться датчиком псевдослучайных чисел, имеющимся в составе стандартной библиотеки языка программирования C. Вторая и третьи компоненты должны вычисляться по формулам \eqref{secondcomp} и \eqref{thirdcomp} соответственно.
		\begin{gather}
			x_2 ~=~ a x_1^2 \label{secondcomp} \\
			x_3 ~=~ \sin(x_1) - \ln(x_1) \label{thirdcomp}
		\end{gather}

		Здесь $x_1$ - значение первой компоненты генерируемой реализации случайного вектора X.

		При расчете матрицы ковариации должно использоваться не менее 1000-и реализаций случайного вектора X;

	\item Усреднение элементов матрицы.

		Усреднение элементов матрицы выполняется путем наложения на каждый элемент матрицы и окрестность элемента матрицы $M$ такой, что:
		\begin{gather*}
			M ~=~ \frac{1}{9}
				\begin{pmatrix}
					1 & 1 & 1 \\ 
					1 & 1 & 1 \\ 
					1 & 1 & 1
				\end{pmatrix}
		\end{gather*}

		Центральный элемент матрицы $M$ должен быть совмещен с текущим обрабатываемым элементом исходной матрицы.

		Исходная матрица должна быть размером не менее 20-ти строк и 20-ти столбцов. Ее размеры должны указываться оператором в процессе выполнения программы;

	\item Дискретное двухмерное вейвлет-преобразование по базису Хаара.

		Дискретное двухмерное вейвлет-преобразование матрицы $M$ по базису Хаара выполняется по формулам \eqref{haarAll}, \eqref{haarA}, \eqref{haarDh}, \eqref{haarDv} и \eqref{haarDd}. Результатом преобразования являются матрицы $A$, $Dh$, $Dv$ и $Dd$, содержащие, соответственно, огрубление матрицы $M$ и уточняющие коэффициенты по горизонтали, вертикали и диагонали. Количество строк и столбцов в матрицах $A$, $Dh$, $Dv$ и $Dd$ вдвое меньше, чем в матрице $M$.
		\begin{gather}
			A, Dh, Dv, Dd ~=~ haar(M) \label{haarAll} \\
			a_{ij} ~=~ \frac{m_{2i~2j} + m_{2i + 1 ~ 2j} + m_{2i ~ 2j + 1} + m_{2i + 1 ~ 2j + 1}}{2} \label{haarA} \\
			dh_{ij} ~=~ \frac{m_{2i~2j} + m_{2i + 1 ~ 2j} - m_{2i ~ 2j + 1} - m_{2i + 1 ~ 2j + 1}}{2} \label{haarDh} \\
			dv_{ij} ~=~ \frac{m_{2i~2j} - m_{2i + 1 ~ 2j} + m_{2i ~ 2j + 1} - m_{2i + 1 ~ 2j + 1}}{2} \label{haarDv} \\
			dd_{ij} ~=~ \frac{m_{2i~2j} - m_{2i + 1 ~ 2j} - m_{2i ~ 2j + 1} + m_{2i + 1 ~ 2j + 1}}{2} \label{haarDd} \\
			m_{ij} \in M; ~ a_{ij} \in A; ~ dh_{ij} \in Dh; ~ dv_{ij} \in Dv; ~ dd_{ij} \in Dd \notag
		\end{gather}

		Для упрощения вычислений, рекомендуется обрабатывать только квадратные матрицы $M$ с четным размером стороны. Размер стороны матрицы $M$ должен быть не менее 20-и;

	\item Обратное дискретное двухмерное вейвлет-преобразование по базису Хаара.

		Обратное дискретное двухмерное вейвлет-преобразование по базису Хаара выполняется по формулам \eqref{rhaarAll}, \eqref{rhaarIJ}, \eqref{rhaarI1J}, \eqref{rhaarIJ1} и \eqref{rhaarI1J1}. Результатом преобразования является матрица $M$, количество строк и столбцов в которой в два раза больше, чем количество строк и столбцов в матрицах $A$, $Dh$, $Dv$ и $Dd$. Матрицы $A$, $Dh$, $Dv$ и $Dd$ должны иметь одинаковые размеры.
		\begin{gather}
			M ~=~ haar^{-1}(A, Dh, Dv, Dd) \label{rhaarAll} \\
			m_{2i ~ 2j} ~=~ \frac{a_{ij} + dh_{ij} + dv_{ij} + dd_{ij}}{2} \label{rhaarIJ} \\
			m_{2i + 1 ~ 2j} ~=~ \frac{a_{ij} + dh_{ij} - dv_{ij} - dd_{ij}}{2} \label{rhaarI1J} \\
			m_{2i ~ 2j + 1} ~=~ \frac{a_{ij} - dh_{ij} + dv_{ij} - dd_{ij}}{2} \label{rhaarIJ1} \\
			m_{2i + 1 ~ 2j + 1} ~=~ \frac{a_{ij} - dh_{ij} - dv_{ij} + dd_{ij}}{2} \label{rhaarI1J1} \\
			m_{ij} \in M; ~ a_{ij} \in A; ~ dh_{ij} \in Dh; ~ dv_{ij} \in Dv; ~ dd_{ij} \in Dd \notag
		\end{gather}

		Для упрощения вычислений, рекомендуется обрабатывать только квадратные матрицы $A$, $Dh$, $Dv$ и $Dd$. Размер стороны перечисленных матриц должен быть не менее 10-и;

	\item Сортировка методом слияния.

		Сортировка методом слияния предполагает разделение массива $M$ на $N$ частей, сортировку каждой из частей (возможно, рекурсивно - методом слияния), и дальнейшее слияние отсортированных частей с сохранением порядка.

		Массив $M$ должен содержать не менее 5000 элементов, при этом количество элементов в массиве должно указываться оператором программы в процессе ее выполнения. Число $N$ должно быть не менее 5 для исходного массива $M$ и произвольным для его частей, в случае использования рекурсии;

	\item Суммирование больших целых положительных чисел.

		Порядок обоих слагаемых должен быть не меньше 8192-х бит.

\end{enumerate}

Задания повышенной сложности:

\begin{enumerate}

	\item Многослойный перцептрон.

		Необходимо реализовать многослойный (количество слоев можно выбрать заранее) перцептрон, обучаемый в пакетном режиме по алгоритму градиентного спуска с использованием метода обратного распространения ошибки. Распараллеливанию подлежат процесс обучения перцептрона и процесс прогона его над тестовой выборкой;

	\item Поиск контуров на бинарном изображении.

		На бинарном (черно-белом) изображении необходимо отыскать и отметить (например, третьим цветом) контуры всех белых объектов;
	
	\item Построение интерференционной картины от двух источников волн.

		Имеется трехмерное Декартово пространство, в две точки которого помещены источники излучения. Необходимо построить интерференционную картину в наборе точек, расположенных на плоскости на некотором удалении от каждого из источников.
		
		Длина волны (одинаковая для обоих источников), амплитуды волн, координаты источников, координаты точек, в которых рассчитывается интерференционная картина, известны и вводятся оператором в процессе выполнения программы.

		Результирующую интерферограмму необходимо визуализировать, сохранив ее в виде изображения в оттенках серого, предварительно смаштабировав энергию результирующей волны в диапазон $[0,~255]$;
	
	\item Компилятор языка программирования Brainfork.

		Компилятор должен генерировать на выходе исполняемый файл формата ELF, предназначенный для запуска в \gl. При этом допускается компиляция программ в код на языке программирования C (C++) с последующей компиляцией результирующего кода в бинарный код сторонним компилятором.
		
		Результатом компиляции инструкции \verb|Y| может стать системный вызов fork().

		Распараллеливанию подлежит компиляция отдельных участков программы;
	
	\item Решение системы линейных уравнений методом Крамера.

		Количество уравнений в системе должно быть не менее 20-ти. Демонстрация выполнения задания должна проводится на совместной невырожденной системе линейных уравнений, однако программа должна определять и корректно обрабатывать случаи несовместной или (и) вырожденной системы.

\end{enumerate}



	\lr{Многопроцессорные системы с общей памятью. Взаимодействие процессов через разделяемую память. \\ Стандарт OpenMP.} % OpenMP
	\lr{Многопроцессорные системы с распределенной памятью. Вычислительные кластеры. Взаимодействие процессов путем обмена сообщениями. \\ Стандарт MPI (Message Passing Interface). Библиотека OpenMPI.} % OpenMPI
	\lr{Свободные программные пакеты, предназначенные для организации вычислительных кластеров и управления ими. \\ Библиотека ROOT / Proof.} % ROOT / Proof
	\lr{Языки параллельного программирования. \\ Язык программирования Erlang.} % Erlang

	\clearpage
	\addcontentsline{toc}{section}{\hskip 16pt Библиография}
	\bibliographystyle{gost780u}
	\bibliography{various/biblio}

\end{document}

