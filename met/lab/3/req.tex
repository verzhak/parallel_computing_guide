
Отчет по лабораторной работе должен содержать:

\begin{itemize}

	\item Титульный лист, оформленный в соответствии с требованиями преподавателя, кафедры и университета и действующих государственных стандартов по оформлению титульных листов отчетов по лабораторным работам;
	
	\item Задание к лабораторной работе;

	\item Исходный код главной программы. Исходный код ядер (главных функций) вычислительных потоков.

	Программа должна быть написана на языке программирования C;

	\item Снимок экрана, иллюстрирующий процесс компиляции программы;

	\item Снимок экрана, иллюстрирующий запуск программы на наборе данных, предполагающем небольшую временную сложность вычислений, и результаты выполнения ручных расчетов для данного набора.

	Для ручного расчета разрешается применять специализированное программное обеспечение - Octave, Sage, Maxima и прочее;

	\item Снимки экрана, иллюстрирующие, как минимум, три запуска программы на наборах данных, соответствующих заданию к лабораторной работе;

	\item Вывод.

	Вывод по выполнению лабораторной работы должен содержать, кроме всего прочего, заключение об эффективности (временном выигрыше или отсутствии такового) распараллеливания реализуемого алгоритма обработки исходных данных.

\end{itemize}

