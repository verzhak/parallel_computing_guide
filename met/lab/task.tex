
\begin{enumerate}

	\item Умножение матриц.
	
		Вычислить матрицу $R$ такую, что $R ~=~ AB$. Здесь $A$ и $B$ матрицы размером не менее 20-ти строк и 20-ти столбцов каждая. Размеры матриц $A$ и $B$ должны указываться оператором программы в процессе ее выполнения;

	\item Решение системы линейных уравнений методом Гаусса.

		Количество уравнений в системе должно быть не менее 20-ти. Демонстрация выполнения задания должна проводится на совместной невырожденной системе линейных уравнений, однако программа должна определять и корректно обрабатывать случаи несовместной или (и) вырожденной системы;

	\item Расчет определителя квадратной матрицы.

		Количество строк в матрице должно быть не менее 7-и;

	\item Расчет определенного интеграла функции от одной переменной.

		Пределы интегрирования должны указываться оператором программы в процессе ее выполнения. Интегрируемую функцию следует реализовать как отдельную подпрограмму, принимающую единственный параметр типа \verb|double| - значение переменной, и возвращающую значение типа \verb|double| - результат вычисления функции. Не допускается программная реализация неопределенного интеграла от интегрируемой функции;

	\item Расчет матрицы ковариации случайного вектора размерности 3.

		Первая компонента реализации случайного вектора должна генерироваться датчиком псевдослучайных чисел, имеющимся в составе стандартной библиотеки языка программирования C. Вторая и третьи компоненты должны вычисляться по формулам \eqref{secondcomp} и \eqref{thirdcomp} соответственно.
		\begin{gather}
			x_2 ~=~ a x_1^2 \label{secondcomp} \\
			x_3 ~=~ \sin(x_1) - \ln(x_1) \label{thirdcomp}
		\end{gather}

		Здесь $x_1$ - значение первой компоненты генерируемой реализации случайного вектора X.

		При расчете матрицы ковариации должно использоваться не менее 1000-и реализаций случайного вектора X;

	\item Усреднение элементов матрицы.

		Усреднение элементов матрицы выполняется путем наложения на каждый элемент матрицы и окрестность элемента матрицы $M$ такой, что:
		\begin{gather*}
			M ~=~ \frac{1}{9}
				\begin{pmatrix}
					1 & 1 & 1 \\ 
					1 & 1 & 1 \\ 
					1 & 1 & 1
				\end{pmatrix}
		\end{gather*}

		Центральный элемент матрицы $M$ должен быть совмещен с текущим обрабатываемым элементом исходной матрицы.

		Исходная матрица должна быть размером не менее 20-ти строк и 20-ти столбцов. Ее размеры должны указываться оператором в процессе выполнения программы;

	\item Дискретное двухмерное вейвлет-преобразование по базису Хаара.

		Дискретное двухмерное вейвлет-преобразование матрицы $M$ по базису Хаара выполняется по формулам \eqref{haarAll}, \eqref{haarA}, \eqref{haarDh}, \eqref{haarDv} и \eqref{haarDd}. Результатом преобразования являются матрицы $A$, $Dh$, $Dv$ и $Dd$, содержащие, соответственно, огрубление матрицы $M$ и уточняющие коэффициенты по горизонтали, вертикали и диагонали. Количество строк и столбцов в матрицах $A$, $Dh$, $Dv$ и $Dd$ вдвое меньше, чем в матрице $M$.
		\begin{gather}
			A, Dh, Dv, Dd ~=~ haar(M) \label{haarAll} \\
			a_{ij} ~=~ \frac{m_{2i~2j} + m_{2i + 1 ~ 2j} + m_{2i ~ 2j + 1} + m_{2i + 1 ~ 2j + 1}}{2} \label{haarA} \\
			dh_{ij} ~=~ \frac{m_{2i~2j} + m_{2i + 1 ~ 2j} - m_{2i ~ 2j + 1} - m_{2i + 1 ~ 2j + 1}}{2} \label{haarDh} \\
			dv_{ij} ~=~ \frac{m_{2i~2j} - m_{2i + 1 ~ 2j} + m_{2i ~ 2j + 1} - m_{2i + 1 ~ 2j + 1}}{2} \label{haarDv} \\
			dd_{ij} ~=~ \frac{m_{2i~2j} - m_{2i + 1 ~ 2j} - m_{2i ~ 2j + 1} + m_{2i + 1 ~ 2j + 1}}{2} \label{haarDd} \\
			m_{ij} \in M; ~ a_{ij} \in A; ~ dh_{ij} \in Dh; ~ dv_{ij} \in Dv; ~ dd_{ij} \in Dd \notag
		\end{gather}

		Для упрощения вычислений, рекомендуется обрабатывать только квадратные матрицы $M$ с четным размером стороны. Размер стороны матрицы $M$ должен быть не менее 20-и;

	\item Обратное дискретное двухмерное вейвлет-преобразование по базису Хаара.

		Обратное дискретное двухмерное вейвлет-преобразование по базису Хаара выполняется по формулам \eqref{rhaarAll}, \eqref{rhaarIJ}, \eqref{rhaarI1J}, \eqref{rhaarIJ1} и \eqref{rhaarI1J1}. Результатом преобразования является матрица $M$, количество строк и столбцов в которой в два раза больше, чем количество строк и столбцов в матрицах $A$, $Dh$, $Dv$ и $Dd$. Матрицы $A$, $Dh$, $Dv$ и $Dd$ должны иметь одинаковые размеры.
		\begin{gather}
			M ~=~ haar^{-1}(A, Dh, Dv, Dd) \label{rhaarAll} \\
			m_{2i ~ 2j} ~=~ \frac{a_{ij} + dh_{ij} + dv_{ij} + dd_{ij}}{2} \label{rhaarIJ} \\
			m_{2i + 1 ~ 2j} ~=~ \frac{a_{ij} + dh_{ij} - dv_{ij} - dd_{ij}}{2} \label{rhaarI1J} \\
			m_{2i ~ 2j + 1} ~=~ \frac{a_{ij} - dh_{ij} + dv_{ij} - dd_{ij}}{2} \label{rhaarIJ1} \\
			m_{2i + 1 ~ 2j + 1} ~=~ \frac{a_{ij} - dh_{ij} - dv_{ij} + dd_{ij}}{2} \label{rhaarI1J1} \\
			m_{ij} \in M; ~ a_{ij} \in A; ~ dh_{ij} \in Dh; ~ dv_{ij} \in Dv; ~ dd_{ij} \in Dd \notag
		\end{gather}

		Для упрощения вычислений, рекомендуется обрабатывать только квадратные матрицы $A$, $Dh$, $Dv$ и $Dd$. Размер стороны перечисленных матриц должен быть не менее 10-и;

	\item Сортировка методом слияния.

		Сортировка методом слияния предполагает разделение массива $M$ на $N$ частей, сортировку каждой из частей (возможно, рекурсивно - методом слияния), и дальнейшее слияние отсортированных частей с сохранением порядка.

		Массив $M$ должен содержать не менее 5000 элементов, при этом количество элементов в массиве должно указываться оператором программы в процессе ее выполнения. Число $N$ должно быть не менее 5 для исходного массива $M$ и произвольным для его частей, в случае использования рекурсии;

	\item Суммирование больших целых положительных чисел.

		Порядок обоих слагаемых должен быть не меньше 8192-х бит.

\end{enumerate}

Задания повышенной сложности:

\begin{enumerate}

	\item Многослойный перцептрон.

		Необходимо реализовать многослойный (количество слоев можно выбрать заранее) перцептрон, обучаемый в пакетном режиме по алгоритму градиентного спуска с использованием метода обратного распространения ошибки. Распараллеливанию подлежат процесс обучения перцептрона и процесс прогона его над тестовой выборкой;

	\item Поиск контуров на бинарном изображении.

		На бинарном (черно-белом) изображении необходимо отыскать и отметить (например, третьим цветом) контуры всех белых объектов;
	
	\item Построение интерференционной картины от двух источников волн.

		Имеется трехмерное Декартово пространство, в две точки которого помещены источники излучения. Необходимо построить интерференционную картину в наборе точек, расположенных на плоскости на некотором удалении от каждого из источников.
		
		Длина волны (одинаковая для обоих источников), амплитуды волн, координаты источников, координаты точек, в которых рассчитывается интерференционная картина, известны и вводятся оператором в процессе выполнения программы.

		Результирующую интерферограмму необходимо визуализировать, сохранив ее в виде изображения в оттенках серого, предварительно смаштабировав энергию результирующей волны в диапазон $[0,~255]$;
	
	\item Компилятор языка программирования Brainfork.

		Компилятор должен генерировать на выходе исполняемый файл формата ELF, предназначенный для запуска в \gl. При этом допускается компиляция программ в код на языке программирования C (C++) с последующей компиляцией результирующего кода в бинарный код сторонним компилятором.
		
		Результатом компиляции инструкции \verb|Y| может стать системный вызов fork().

		Распараллеливанию подлежит компиляция отдельных участков программы;
	
	\item Решение системы линейных уравнений методом Крамера.

		Количество уравнений в системе должно быть не менее 20-ти. Демонстрация выполнения задания должна проводится на совместной невырожденной системе линейных уравнений, однако программа должна определять и корректно обрабатывать случаи несовместной или (и) вырожденной системы.

\end{enumerate}

