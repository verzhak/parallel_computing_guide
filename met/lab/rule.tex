
Для успешного выполнения и сдачи лабораторных работ в отведенные для этого учебной программой сроки студент должен соблюдать следующие правила:

\begin{enumerate}

	\item Задания к лабораторным работам:

	\begin{enumerate}
		
		\item Каждая бригада студентов получает индивидуальное задание, номер которого равен увеличенному на единицу остатку от деления на 10 номера бригады студентов в журнале лабораторных работ;
		\item Студенты с хорошей успеваемостью, желающие выполнить более сложное задание, могут выбрать его себе из списка заданий повышенной сложности. Задания повышенной сложности отмечены символом *. В случае такового выбора студент освобождается от выполнения задания, выданного его бригаде. Не допускается бригадное выполнение задания повышенной сложности;
		\item Единожды полученные задания выполняются бригадами студентов / студентами на каждой из лабораторных работ каждый раз в установленных методическими указаниями (МУ) к лабораторной работе рамках с использованием программных средств и языка программирования, указанных в означенных МУ;
		\item Допускается отказ от выполнения студентом задания повышенной сложности. В этом случае он обязан выполнять задание своей бригады;
		\item Не допускается замена заданий;
		\item Качество заданий, полнота их описания и прочие вопросы, относящиеся к заданиям и определенные преподавателем, не обсуждаются. Отсутствие теоретических рекомендаций по выполнению задания предполагает самостоятельное изучение вопроса;

	\end{enumerate}

	\item Количество выполняемых лабораторных работ устанавливается преподавателем {\bf в начале семестра} и не может быть пересмотрено по просьбе студентов группы;

	\item Схема проведения лабораторной работы:

	\begin{enumerate}

		\item Допуск студентов к выполнению лабораторной работы;
		\item Выполнение студентами лабораторной работы;\label{list:exec}
		\item Демонстрация преподавателю результатов выполнения лабораторной работы;
		\item Оценка преподавателем качества выполнения лабораторной работы. В случае неудовлетворительной оценки выполняется переход к пункту \ref{list:exec};
		\item Подготовка {\bf каждым} студентом индивидуального отчета по лабораторной работе;\label{list:defend}
		\item Допуск студентов к защите лабораторной работы;
		\item Защита студентами лабораторной работы. В случае провальной защиты лабораторной работы выполняется переход к пункту \ref{list:exec} или к пункту \ref{list:defend} по выбору преподавателя;

	\end{enumerate}

	\item Выполнение лабораторных работ:

	\begin{enumerate}

		\item К выполнению лабораторной работы допускаются студенты, успешно выполнившие все предыдущие лабораторные работы и успешно защитившие все предыдущие, кроме, возможно, последней, лабораторные работы;
		\item В случае, если в бригаде до выполнения лабораторной работы допущены лишь часть студентов, то студенты данной бригады, не допущенные к лабораторной работе, должны будут {\bf самостоятельно} выполнить практическую часть лабораторной работы и продемонстрировать результаты ее выполнения преподавателю;
		\item После того, как лабораторная работа была выполнена, результаты ее выполнения должны быть продемонстрированы преподавателю;

	\end{enumerate}

	\item Защита лабораторных работ:

	\begin{enumerate}

		\item Для допуска к защите лабораторных работ {\bf каждый} студент группы должен самостоятельно подготовить уникальный отчет;
		\item Для допуска к защите лабораторных работ студент должен защитить все предыдущие лабораторные работы;
		\item Отчет должен соответствовать требованиям, перечисленным в настоящей МУ, и личным требованиям преподавателя;
		\item За преподавателем остается право не допустить студента до защиты лабораторной работы или аннулировать данный допуск в следующих случаях:

		\begin{enumerate}

			\item Студент не выполнил лабораторную работу или не продемонстрировал преподавателю в достаточной степени\footnote{<<Достаточность>> здесь и везде определяется преподователем.} работоспособность разработанного им программного решения;
			\item У преподавателя имеются сомнения в самостоятельном выполнении студентом лабораторной работы;
			\item У преподавателя имеются сомнения в самостоятельной подготовке студентом отчета по лабораторной работе;
			\item Во всех прочих случаях, когда преподаватель по той или иной причине не считает возможным допустить студента до защиты лабораторной работы;

		\end{enumerate}

		\item Для успешной защиты лабораторной работы студент должен в достаточной степени полно и правильно ответить на все вопросы преподавателя. Преподаватель имеет право задавать любой вопрос по отчету, методическим указаниям к защищаемой или предыдущим лабораторным работам, лекциям. Для решения спорных вопросов относительно присутствия того или иного материала в прочитанных на данный момент лекциях следует обращаться к преподавателям, читавшим (читающим) данные курсы. Конспект студента не является аргументом в споре о правильности того или иного решения преподавателя;

	\end{enumerate}

	\item Отчет по лабораторной работе:

	\begin{enumerate}

		\item Отчет по лабораторной работе должен быть представлен в печатном виде;
		\item Отчет по лабораторной работе должен быть выполнен в соответствии с требованиями преподавателя, кафедры, университета и в соответствии с действующими на данный момент государственными стандартами;
		\item Отчет должен быть подготовлен, по возможности, с помощью свободного программного обеспечения, к коему относятся офисный пакет LibreOffice, текстовый процессор AbiWord, система подготовки текстов к публикации \LaTeX~и прочее программное обеспечение;
		\item Иллюстрации в отчетах должны быть выполнены, по возможности, с помощью свободного программного обеспечения, к коему относятся редакторы диаграмм Dia и LibreOffice Draw, векторные графические редакторы Inkscape и Sk1, растровые графические редакторы Gimp, Krita и Nip2 и прочее программное обеспечение;
		\item Факт подготовки отчета с помощью проприетарного программного обеспечения (например, с помощью операционной системы Windows или (и) с помощью офисного пакета Microsoft Office) является отягчающим обстоятельством при принятия решения о допуске студента до защиты лабораторной работы в случае низкого качества отчета;
		\item Не допускаются до защиты отчеты, выполненные (написанные и проиллюстрированные) полностью или частично от руки;
		\item Факт наличия большого количества исправлений, внесенных от руки, орфографических и пунктационных ошибок в отчете является отягчающим обстоятельством при принятия решения о допуске студента до защиты лабораторной работы в случае низкого качества отчета;

	\end{enumerate}

	\item За преподавателем остается право не аргументировать любое свое решение;
	\item Настоящие правила не обсуждаются;
	\item Студент считается ознакомленным с правилами после того, как им было начато выполнение первой лабораторной работы;
	\item Настоящие правила могут быть {\bf дополнены} преподавателем положениями, не отменяющими полностью или частично перечисленные положения. Преподаватель обязан незамедлительно (на текущем или следующем занятии) довести до сведения студентов изменения в правилах.

\end{enumerate}

