
В качестве демонстрационного примера рассмотрим моделирование показательного распределения с помощью датчика псевдослучайных чисел, равномерно распределенных на диапазоне $[0 ;~ 1]$.

Плотность показательного распределения вычисляется по формуле \eqref{exp:distrib}.
\begin{gather}
	f(x) = \lambda e^{- \lambda x} \label{exp:distrib} \\
	\lambda \in \mathbb{R} \notag
\end{gather}

Математическое ожидание случайной величины с показательным распределением равно $\dfrac{1}{\lambda}$. Дисперсия случайной величины с показательным распределением равна $\dfrac{1}{\lambda ^ 2}$.

Генерирование реализации $\xi$ псевдослучайной величины с показательным распределением происходит по формуле \eqref{exp:gen}.
\begin{gather}
	\xi = \frac{- \ln(\theta)}{\lambda} \label{exp:gen} \\
	\theta \in \mathbb{R} \notag
\end{gather}

В формуле \eqref{exp:gen} значение $\theta$ суть есть реализация псевдослучайной величины, распределенной равномерно на диапазоне $[0 ;~ 1]$.

Исходный код демонстрационной программы приведен в листинге \ref{listing:erlang:src}.

Рисунком \ref{image:erlang:amv-1} проиллюстрированы процесс запуска виртуальной машины на вычислительном узле <<amv\_1@127.0.0.1>> и процесс компиляции на означенном узле разработанного программного модуля. Рисунком \ref{image:erlang:amv-2} проиллюстрированы процесс запуска виртуальной машины на вычислительном узле <<amv\_2@127.0.0.1>> и процесс компиляции на означенном узле разработанного программного модуля. Наконец, на рисунке \ref{image:erlang:run} приведены результаты нескольких тестовых запусков моделирования показательного распределения.

%\mysource{../src/demo_2/erlang/demo.erl}{erlang:src}{Демонстрационная программа} % TODO раскомментировать
%\mimage{erlang:amv-1}{TODO}{Запуск виртуальной машины на вычислительном узле <<amv\_1@127.0.0.1>>, компиляция разработанного программного модуля}{}
%\mimage{erlang:amv-2}{TODO}{Запуск виртуальной машины на вычислительном узле <<amv\_2@127.0.0.1>>, компиляция разработанного программного модуля}{}
%\mimage{erlang:run}{TODO}{Результаты нескольких тестовых запусков моделирования показательного распределения}{}

