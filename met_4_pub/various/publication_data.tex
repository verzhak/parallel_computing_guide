
\thispagestyle{empty}

{\raggedright УДК TODO}

Параллельные и распределенные вычисления: методические указания к лабораторным работам / Рязан. гос. радиотехн. ун-т; сост. М.В. Акинин, В.А. Саблина. Рязань, 2013. 62 с. % TODO Сколько, сколько страниц?

Содержат материалы для выполнения лабораторных работ по курсу <<Параллельное программирование>>. Курс лабораторных работ включает примеры параллельных реализаций алгоритмов численного моделирования и задания для самостоятельного выполнения. Курс предназначен для развития практических навыков организации параллельных и распределенных вычислений при решении задач моделирования и обработки больших объемов данных. Для обучения требуется знание основ программирования на языке программирования C.

Предназначены студентам специальности 010503 бакалаврам направления 010500 <<Математическое обеспечение и администрирование информационных систем>>. Могут использоваться аспирантами и специалистами, желающими развить навыки организации параллельных и распределенных вычислений.

Табл. 2. Ил. 11. Библиогр.: 8 назв.

\bigskip
\textit{C, OpenMP, OpenMPI, OpenCL, Erlang}
\bigskip

Печатается по решению редакционно-издательского совета Рязанского государственного радиотехнического университета.

Рецензент: кафедра ЭВМ РГРТУ (зав. кафедрой проф. В.К. Злобин)

\vspace{0.5cm}

\begin{center}
Параллельные и распределенные вычисления

\bigskip

Составители: Акинин Максим Викторович, Саблина Виктория Александровна

\vspace{0.3cm}

Редактор Н.А. Орлова \linebreak
Корректор М.Е. Цветкова \linebreak
Подписано в печать TODO.TODO.13. Формат бумаги 60х84 1/16. \linebreak
Бумага газетная. Печать трафаретная. Усл. печ. л. TODO. \linebreak
Тираж 25 экз. Заказ ~~~~~~~. \linebreak
Рязанский государственный радиотехнический университет. \linebreak
390005, Рязань, ул. Гагарина, 59/1. \linebreak
Редакционно-издательский центр РГРТУ.
\end{center}

\clearpage

\setcounter{page}{3}

