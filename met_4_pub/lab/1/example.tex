
В качестве демонстрационного примера рассмотрим моделирование распределения Стьюдента с помощью датчика псевдослучайных чисел, равномерно распределенных на диапазоне $[0 ;~ 1)$.

Плотность распределения Стьюдента вычисляется по формуле \eqref{student:distrib}.
\begin{gather}
	f(x) = \frac{\Gamma((N + 1) / 2)}{\sqrt{\pi N} ~ \Gamma(N / 2)} \left( 1 + \frac{x^2}{N} \right) ^{- (N + 1) / 2} \label{student:distrib} \\
	N \in \mathbb{N} ;~ N > 2 \notag 
\end{gather}

Здесь $N$ суть число степеней свободы распределения Стьюдента. 

В формуле \eqref{student:distrib} функция $\Gamma(t)$ суть гамма - функция, вычисляемая по формуле \eqref{gamma:func}.
\begin{gather}
	\Gamma(t) = \int \limits_0^{\infty} x^{t - 1} e^{-x} \, dx \label{gamma:func}
\end{gather}

Математическое ожидание случайной величины, распределенной по Стьюденту, равно 0. Ее дисперсия, в случае, если число степеней свободы больше двух, равна $\dfrac{N}{N - 2}$.

Генерирование реализации псевдослучайной величины, распределенной по Стьюденту с $N$ степенями свободы ($N > 2$), происходит в несколько шагов:

\begin{enumerate}

	\item генерирование $(N + 1)$ реализаций $\{\xi_1, \xi_2, ..., \xi_{N + 1}\}$ псевдослучайных величин, распределенных нормально на диапазоне \linebreak $(-1 ;~ 1)$.

	Для генерирования реализации нормально распределенной псевдослучайной величины с помощью датчика псевдослучайных равномерно распределенных чисел используется центральная предельная теорема: сумма реализаций бесконечного числа одинаково распределенных независимых случайных величин распределена нормально. В данном случае, бесконечная сумма заменяется суммой 12-ти реализаций псевдослучайной величины, равномерно распределенной на диапазоне $(-1 ;~ 1)$;
	
	\item генерирование реализации $x$ псевдослучайной величины, распределенной по Стьюденту.

	Генерирование реализации $x$ псевдослучайной величины, распределенной по Стьюденту, осуществляется по формуле \eqref{student:x}.
	\begin{gather}
		x = \frac{\xi_{N + 1}}{\sqrt{\frac{1}{n}\sum \limits_{i = 1}^N \xi_i^2}} \label{student:x}
	\end{gather}

\end{enumerate}

Исходный код демонстрационной программы доступен по ссылке \cite{code} (файл openmp/main.c).

Рисунком \ref{image:openmp:build} проиллюстрирован процесс сборки демонстрационной программы. На рисунках \ref{image:openmp:demo:1:res} и \ref{image:openmp:demo:1:htop} приведены результаты тестового запуска демонстрационной программы, в ходе которого поставленное задание выполнили два вычислительных потока.

\mimage{openmp:build}{15}{Сборка демонстрационной программы}{width=0.75\textwidth}
\mimage{openmp:demo:1:res}{17}{Выполнение вычислений двумя потоками (htop)}{width=0.75\textwidth}
\mimage{openmp:demo:1:htop}{9}{Выполнение вычислений двумя потоками (вывод программы)}{width=\textwidth}

